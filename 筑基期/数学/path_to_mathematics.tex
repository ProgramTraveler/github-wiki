\documentclass[article]{article}
\usepackage[fontset=ubuntu]{ctex} % 中文显示包
% 超链接
\usepackage{hyperref}
\hypersetup{hidelinks,
colorlinks = true,
allcolors = blue,
pdfstartview = Fit,
breaklinks = true
}

\usepackage{amsmath}
% title
\begin{document}
\title{Path to Mathematics}
\maketitle

\section{符号}
记录自己在阅读数学相关知识时,遇到的不懂的数学符号和它们对应的含义,如果你在阅读的过程中发现存在问题,那么你可以给我留言让我可以弥补错误,thanks bro。
\subsection{\texorpdfstring{$R$}{}}
该符号表示为实数。

\subsection{\texorpdfstring{$R^n$}{}}
实 $n$-维向量,也就是 $n*1$ 的矩阵。

在我阅读凸优化的相关知识时,我发现有个 $f: R^n \to R$,经过一些查询 \href{https://math.stackexchange.com/questions/1852420/what-does-mathbbrn-to-mathbbrm-mean-and-what-is-mathbbrn}{传送门1.2.1}、\href{https://math.ryerson.ca/~danziger/professor/MTH141/Handouts/vectors.pdf}{传送门1.2.2},该表达式其实也可以视为 $f(R^n) = R$,其中 $f$ 可以看为一个 $1 * n$ 的矩阵 $A$

\subsection{\texorpdfstring{$\text{\textbardbl} Ax -b \text{\textbardbl}_2^2$}{}}
这是在最小二乘中出现的一个表达式,下标 $2$ 表示的是 2 范式,而上标 $2$ 表示的平方。\href{https://math.stackexchange.com/questions/3045899/if-a-has-orthonormal-columns-then-ax2-2-x2-2-why}{传送门1.3.1}

其中,最小二乘的表达式也在这里给出 $$\text{\textbardbl} Ax -b \text{\textbardbl}_2^2 = \displaystyle\sum_{i = 1}^k(a^T_i - bi)^2$$.

\subsection{向量}
如果$a,b,c \in R$,对于 $R^3$ 中的某个向量有 $$(a,b,c) = \begin{bmatrix} a\\ b\\ c \end{bmatrix} = \begin{bmatrix}
    a && b && c
\end{bmatrix}^T$$

\section{名词}
这个位置是一些数学名词的解释。是来自 wiki 的,放在这里仅仅只是为了省去在浏览器打字的时间。
\subsection{凸优化}
\href{https://zh.wikipedia.org/wiki/%E5%87%B8%E5%84%AA%E5%8C%96}{wiki 凸优化} 
\begin{quotation} % 引用
    优化问题的分水岭不是线性和非线性,而是凸性和非凸性
\end{quotation}
    
\subsection{最小二乘}
\href{https://zh.wikipedia.org/wiki/%E6%9C%80%E5%B0%8F%E4%BA%8C%E4%B9%98%E6%B3%95}{wiki 最小二乘}

判断一个优化问题是否为最小二乘问题非常简单,只需要检验目标函数是否为二次函数,然后检验此二次函数是否半正定。TODO: 什么是半正定?

\subsection{线性规划}
其目标函数和所有的约束函数均为线性函数。
\end{document}